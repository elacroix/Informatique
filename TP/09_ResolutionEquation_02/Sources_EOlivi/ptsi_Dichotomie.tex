%\documentclass[preview,varwidth=11cm]{standalone}

\documentclass[11pt,a4paper]{article}
\usepackage{vmargin}
\setmarginsrb{1cm}{1cm}{1cm}{1cm}{0cm}{0cm}{0cm}{0cm}

\usepackage[utf8]{inputenc}
%%%%\usepackage[latin1]{inputenc} 

\usepackage[T1]{fontenc}
\usepackage[francais]{babel}
\usepackage{enumerate}
\usepackage{setspace}

\usepackage{fancybox,slashbox}
\usepackage{amsmath,amsfonts,amssymb,mathrsfs}
\usepackage{multicol}
\usepackage{fourier}
\usepackage{pifont}
\usepackage{array,pstricks,pstricks-add,pst-all,pst-plot}
%%%%\usepackage{MnSymbol}
\usepackage{enumerate}
%%%%\usepackage{minitoc}


\pagestyle{empty}
\setlength{\parindent}{0pt}


\renewcommand{\:}{\textbackslash}

%%raccourci en mode math
\newcommand{\pti}{\scriptstyle}
\newcommand{\ds}{\displaystyle}
\newcommand{\RR}{\mathbb{R}}
\newcommand{\ZZ}{\mathbb{Z}}
\newcommand{\QQ}{\mathbb{Q}}
\newcommand{\NN}{\mathbb{N}}
\newcommand{\CC}{\mathbb{C}}
\newcommand{\KK}{\mathbb{K}}
\newcommand{\equ}{\Longleftrightarrow}
\newcommand{\mcal}{\mathcal}
\newcommand{\ve}{\overrightarrow}
%%\newcommand{\oo}[2][]{\mathrm{o}#1(#2#1)}
\newcommand{\oo}{\mathrm{o}}
\newcommand{\OO}{\mathrm{O}}
\newcommand{\abs}[1]{\left|#1\right|}
\newcommand{\ens}[1]{\left\{#1\right\}}
%%suite serie
\newcommand{\suite}[1]{\left(#1_n\right)_n}
\newcommand{\serie}[1]{\sum #1_n}
%%fonction
\newcommand{\fff}[5]{
\begin{array}{rcl}
#1 : #2 & \longrightarrow & #3\\
#4 & \mapsto & #1(#4)=#5
\end{array}
}
%%%% Les 'rm'
\newcommand{\vect}{{\rm vect}}
\newcommand{\sh}{{\rm sh}}
\newcommand{\ch}{{\rm ch}}
\renewcommand{\th}{{\rm th}}
\newcommand{\ii}{{\rm i}}
\newcommand{\ee}{{\rm e}}
\newcommand{\dd}{\,{\rm d}}
%%
%%intervalle d'entiers
\newcommand{\oc}{[\hspace{-.3ex}[}
\newcommand{\fc}{]\hspace{-.3ex}]}
%%
%%espace horizontale
\newcommand{\eh}[1]{\rule{#1}{0pt}}
%%
%%deux coordonnées ou combinaison
\newcommand{\coo}[2]{
\left(\hspace{-1ex}\begin{array}{c} #1 \\ #2	
\end{array}\hspace{-1ex}\right)}
%%
%%trois coordonnées
\newcommand{\cooT}[3]{
\left(\hspace{-1ex}\begin{array}{c} #1 \\ #2 \\ #3	
\end{array}\hspace{-1ex}\right)}
%%
%%quatre coordonnées
\newcommand{\cooQ}[4]{
\left(\hspace{-1ex}\begin{array}{c} #1 \\#2 \\#3 \\ #4	
\end{array}\hspace{-1ex}\right)}
%%
%%environnement pour in-équations, equivalence, etc..
%% lhs & = rhs \\
\newcommand{\equa}{$$\ds\begin{array}{rl}}	
\newcommand{\equaend}{\end{array}$$}
%%
%% 4 réponses possibles sur une même ligne
\newcommand{\qqq}[4]{

\medskip

a.\ #1\hfill b.\ #1\hfill c.\ #1\hfill d.\ #1\hfill }
%%
%%ite
\newcounter{itemm}
\newcommand{\ite}{%%
\bigskip
\refstepcounter{itemm}
\textbf{\theitemm .\ }}
%%
%%razite
\newcommand{\razite}{\setcounter{itemm}{0}}
%%
%%espace phantom:
\newcommand{\fantome}[1]{\rule{0pt}{1ex}\phantom{#1}}
%%
%%exerice
\newcounter{exer}
\newcommand{\exo}[1]{
\refstepcounter{exer}\hbox{}\hrulefill\textbf{Exercice \theexer \ :} \hrulefill\hrulefill\hrulefill\hrulefill(environ #1 pts)\hrulefill\hbox{}
}
%%%%exerice a numero fixe
\newcommand{\exonum}[2]{
\hbox{}\hrulefill\textbf{Exercice #1 \ :} \hrulefill\hrulefill\hrulefill\hrulefill(environ #2 pts)\hrulefill\hbox{}
}
%%
%%razexo
\newcommand{\razexo}{\setcounter{exer}{0}}
%%
%%correction correction2colonnes
\newcommand{\cdcol}[2]{
\begin{minipage}[t]{.65\textwidth}
#1
\end{minipage}
\hfill\vrule\hfill
\begin{minipage}[t]{.3\textwidth}
\textit{\small #2}
\end{minipage}
}
%%des lignes de pointilles
\newcommand{\ppp}[1]{%%
\multido{}{#1}{\makebox[\linewidth]{\dotfill}\\[\parskip]
}}
%%
%%fontsize{n1}{n2} en pt, n1:taille font, n2:taille interligne

%%--------------------------------------------------  defin propr theor preuv corol (5 caractères)
\newlength{\largeurline}
%%définition
\newcommand{\defin}[2]{
\setlength{\largeurline}{\linewidth}
\addtolength{\largeurline}{-6.5ex}
\medskip

\textbf{Définition : #1}\\
\rule{5ex}{0pt}\vrule\hspace{1ex}\begin{minipage}{\largeurline}
#2
\end{minipage}

\medskip
}
%%
%%théorème
\newcommand{\theor}[2]{
\setlength{\largeurline}{\linewidth}
\addtolength{\largeurline}{-6.5ex}
\medskip

\textbf{Théorème : #1}\\
\rule{5.5ex}{0pt}\fbox{\begin{minipage}{\largeurline}
#2
\end{minipage}}

\medskip
}
%%
%%corolaire
\newcommand{\corol}[2]{
\setlength{\largeurline}{\linewidth}
\addtolength{\largeurline}{-6.5ex}
\medskip

\textbf{Corolaire : #1}\\
\rule{5.5ex}{0pt}\fbox{\begin{minipage}{\largeurline}
#2
\end{minipage}}

\medskip
}
%%
%%propriété
\newcommand{\propr}[1]{
\setlength{\largeurline}{\linewidth}
\addtolength{\largeurline}{-6.5ex}
\medskip

\textbf{Propriété : }\\
\rule{5ex}{0pt}\vrule\hspace{1ex}\begin{minipage}{\largeurline}
#1
\end{minipage}

\medskip
}
%%
%%preuve
\newcommand{\preuv}[1]{
\medskip

\textbf{Démonstration : }\\
#1\hfill$\square$

\medskip
}
%%
%%a retenir (a droite)
\newlength{\reteni}
\newcommand{\retenir}[1]{%%
\settowidth{\reteni}{\mbox{#1}}
\addtolength{\reteni}{1ex}
\medskip
\hbox{}\hfill\fbox\exo{\begin{minipage}{\reteni}
\begin{flushright}
#1
\end{flushright}
\end{minipage}}
\medskip
}



\usepackage{color}


\begin{document}

\subsection*{Test intégrale}

Ecrire une fonction \textbf{Intgrl}(f,a,b,n) où :
\begin{itemize}
\item a et b sont les bornes (ordonnées) d'un intervalle,
\item n est le nombre de subdivisions de [a;b]
\item f désigne une fonction définie sur [a;b]
\end{itemize}
qui renvoie la somme de aires des trapèzes.



\subsection*{Exercice : Comparaison}


Soit l'équation \emph{E} : \quad $x\ln(x)=1$ .

\medskip

On admet (si un doute vérifiez-le rapidement) que \emph{E} admet une unique solution $\alpha$ sur $\mathbb R$ et que
$\alpha\in[1;\text{e}]$.

(On prendra e$\approx2.8$)

\medskip

\setlength{\columnseprule}{.2pt}
\begin{multicols}{3}
A.
\begin{enumerate}
\item Complétez la fonction suivante renvoyant une valeur approchée à \texttt{epsilon} près de cette solution par dichotomie.

\medskip

\hspace{-2em}\fbox{
\begin{minipage}{\linewidth}
\ttfamily

def dichotomie(f,epsilon):

\rule{4ex}{0pt} a=\ldots

\rule{4ex}{0pt} b=\ldots

\rule{4ex}{0pt} while abs(\ldots

\rule{8ex}{0pt} c=(a+b)/2

\rule{8ex}{0pt} if f(a)*f(b) < 0 :

\rule{12ex}{0pt} \ldots =c

\rule{8ex}{0pt} else :

\rule{12ex}{0pt} \ldots =c

\rule{4ex}{0pt} return \ldots
\end{minipage}}

\medskip\normalfont

\item Vérifier qu'à 10$^{-6}$ près :

$\alpha\approx 1.763223$ .

\item Modifier cette fonction afin qu'elle affiche le nombre de répétition de la boucle.
\end{enumerate}

\vfill\columnbreak

B.

Modifiez la fonction précédente renvoyant une valeur approchée à \texttt{epsilon} près de cette solution par la méthode des cordes (dite de Lagrange).

\medskip

Avec affichage du nombre de répétition de la boucle.

\medskip

Comparer avec le précédent.

\vfill\columnbreak

C.

On souhaite comparer avec la méthode Newton, mais le test d'arrêt ne permet pas d'obtenir une valeur approchée avec un précision donnée !

\medskip

Nous allons donc nous servir des résultats précédents.

\medskip

\texttt{f} et \texttt{fp} étant des fonctions saisies en python, et correspondant à $f$ et $f'$ alors l'algorithme de Newton est :

\medskip


\fbox{
\begin{minipage}{\linewidth}
\ttfamily

x=2.8

compteur=0 

while abs(x-1.763223)<10**(-6):

\rule{4ex}{0pt} x=\ldots

\rule{4ex}{0pt} compteur=\ldots

print(compteur)

\end{minipage}}

\medskip\normalfont

Comparer

\vfill
\end{multicols}





\subsection*{Exercice : Suite définie implicitement}

Soit $f_n$ une suite de fonction définie sur $\mathbb R$ pour tout $n\in\mathbb N^*$ tel que :

\hfil $f_n(x)=x^n+x^{n-1}+x-1$

\begin{enumerate}
\item Montrer que, pour tout $n\in\mathbb N^*$, l'équation $f_n(x)=0$ possède une unique solution sur $[0;1]$.

\medskip

Notons $u_n$ cette solution.

\item Ecrire un programme Python qui :
\begin{itemize}
\item définie une fonction \texttt{f(x,n)} renvoyant la valeur $f_n(x)$,
\item définie une fonction \texttt{dichotomie(f,n)} renvoyant un valeur à $10^{-3}$ près de $u_n$,
\item calcule et affiche les 100 premières valeurs approchées de
$(u_n)$.
\end{itemize}

\item Conjecturer du comportement de cette suite.
\end{enumerate}



\subsection*{Exercice : Les délices empoisonnées de la méthode de Newton}

\textbf{Partie A}

\medskip

\begin{minipage}{8cm}
On rappelle la méthode de Newton :

\psset{xunit=1.5cm,yunit=1.0cm,algebraic=true,dimen=middle,dotstyle=o,dotsize=3pt 0,linewidth=0.2pt,arrowsize=3pt 2,arrowinset=0.25}
\begin{pspicture*}(-1.,-1.)(4.,5.5)
%\psaxes[labelFontSize=\scriptstyle,xAxis=true,yAxis=true,Dx=1.,Dy=1.,ticksize=-2pt 0,subticks=2]{->}(0,0)(-1.,-1.)(4.,5.5)
\psline{->}(-1.,0)(4.,0)
\psplot[linewidth=1pt]{-1.}{4.}{-1+2.71828^(0.5*x)}
\psplot{-1.}{4.}{-3.2408445351690327+2.2408445351690323*x}
\psplot{.1}{2}{(-0.4294061482447453+1.0304369953425587*x)}
\psset{linestyle=dashed}
\psline(1.4462603202968598,0.)(1.4462603202968598,1.0608739906851175)
\psline(3.,0.)(3.,3.4816890703380645)
%\begin{scriptsize}
\psdots[dotstyle=*](0,0)(3.,0.)(3.,3.4816890703380645)(1.4462603202968598,0.)(1.4462603202968598,1.0608739906851175)(0.4167223713682693,0.)
\rput(0,.3){$\alpha$}
\rput(3,-.3){$x_0$}
\rput(1.4462603202968598,-.3){$x_1$}
\rput(0.4167223713682693,-.3){$x_2$}
%\end{scriptsize}
\end{pspicture*}
\end{minipage}
\hfill
\begin{minipage}{9cm}
Supposons que les fonctions $f$ et $f'$ soient définies en entête du programme,

compléter l'algorithme afin de calculer les 20 premières valeurs de la suite générée par la méthode de Newton :\\


\fbox{\begin{minipage}{\linewidth}

\ttfamily

\rule{0pt}{1.5em}
x = x$_0$\\

pour k allant de 1 à 20 faire :\\

\rule{1cm}{0pt} x =\\

\end{minipage}
}
\end{minipage}\hfill\hbox{}

\bigskip

\textbf{Partie B}

\medskip

Soit la fonction $f$ définie sur $\mathbb R$ par :\quad
$f(x) = x^3 - x^2 - 20x - 12$.

\medskip

Considérons la suite définie par la méthode de Newton en partant de $x_0=2$

Ecrire un programme en Python qui :
\begin{itemize}
\item défini la fonction $f$ ainsi que sa fonction dérivée que vous noterez $fp$,
\item calcule et affiche les valeurs successives de l'algorithme de Newton.
\end{itemize}


\begin{center}
\psset{xunit=1.0cm,yunit=.1cm,
algebraic=true,dimen=middle,dotstyle=*,dotsize=3pt 0,linewidth=0.3pt,arrowsize=3pt 2,arrowinset=0.25}
\begin{pspicture*}(-5.,-55.)(5.,25.)
%\multips(0,-50)(0,10.0){9}{\psline[linestyle=dashed,linecap=1,dash=1.5pt 1.5pt,linewidth=0.4pt,linecolor=lightgray]{c-c}(-5.,0)(5.,0)}
%\multips(-4,0)(2.0,0){6}{\psline[linestyle=dashed,linecap=1,dash=1.5pt 1.5pt,linewidth=0.4pt,linecolor=lightgray]{c-c}(0,-55.)(0,25.)}
\psaxes[labelFontSize=\scriptstyle,xAxis=true,yAxis=true,Dx=2.,Dy=10.,ticksize=-2pt 0,subticks=2]{->}(0,0)(-5.,-55.)(5.,25.)
\psplot[plotpoints=200,linewidth=.8pt]{-5.0}{5.0}{x^(3.0)-x^(2.0)-20.0*x-12.0}
%\psset{linestyle=dotted}
\psplot{-3.}{4.}{(8.-4.*x)/1.}
\psplot{-3.5}{3.}{(-24.-12.*x)/1.}
\psdots(2,-48)(-2,16)
\end{pspicture*}
\end{center}


\subsection*{Exercice : Avec des listes}

Lors d'une expérience on mesure un phénomène numérique au cours du temps et on dresse deux listes (de même longueur) :
\begin{itemize}
\item \texttt{V} : la liste des mesures supposées monotones, 
\item \texttt{T} : la liste des temps (en seconde, dans l'ordre croissant) correspondant à chaque mesure.
\end{itemize}
Exemple : \texttt{T=[0, 2, \ldots]} et \texttt{V=[0.5, 0.55, \ldots]} signifie que 0.5 a été mesuré à 0s, puis la valeur suivante  (0.55) a été prise à 2s etc.

Ecrire une fonction \texttt{solution(valeur,V,T)}, qui renvoie un
encadrement du temps pour lequel les mesures encadrent elles-même la valeur.

Indication :
\begin{itemize}
\item privilégiez la dichotomie,
\item déterminez les indices \texttt{i} et \texttt{j} de cet encadrement,
\item observez que le test d'arrêt de recherche de ces entiers ne dépend pas d'une précision.
\end{itemize}

Tester ce programme avec les listes : \texttt{T=[0, 2, 3, 5, 6, 8, 10]} et \texttt{V=[0.5, 0.55, 0.7, 0.9, 1, 1.5, 1.6]} avec 0.6  pour la valeur.

Modifier votre fonction \texttt{solution} afin de gérer (par un message d'erreur) le cas où il n'y a pas d'encadrement possible car \texttt{valeur} serait incompatible.
\end{document}